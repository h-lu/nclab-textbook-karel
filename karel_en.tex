\documentclass[article,A4,12pt]{llncs}

% Conditional compilation.
% NOTE: If you set fullversionfalse, just compile ONCE so that TOC stays unchanged.
\newif\iffullversion
\fullversiontrue
%\fullversionfalse

\usepackage[T1]{fontenc}
\usepackage{amsmath}
\usepackage{amssymb}
\usepackage{color}
\usepackage{amsfonts}
\usepackage{mathrsfs, bm}

\usepackage{graphicx}
\usepackage{tabularx}
\usepackage{subfig}
\usepackage{epsf,times}
\usepackage{color}
\usepackage{wrapfig}
\usepackage{cases}
\usepackage{multicol}
\usepackage[usenames,dvipsnames]{xcolor}

\usepackage{palatino}

\usepackage[T1]{fontenc}
%\newcommand{\tmname}[1]{\textsc{#1}}
%\newcommand{\tmop}[1]{\ensuremath{\operatorname{#1}}}
%\newcommand{\tmsamp}[1]{\textsf{#1}}
%\newcommand{\tmtextsc}[1]{{\scshape{#1}}}
%\newcommand{\tmtextsl}[1]{{\slshape{#1}}}
%\newcommand{\tmtexttt}[1]{{\ttfamily{#1}}}

\leftmargin=0.0cm
\oddsidemargin=0.5cm
\evensidemargin=0.5cm
\topmargin=0cm
\textwidth=16.0cm
%\textheight=21.5cm
\textheight=20.0cm
\pagestyle{plain}
\setlength{\columnsep}{20pt}

\def\m{\mathbf{m}}
\def\H{\mathbf{H}}
\def\E{\mathbf{E}}
\newcommand{\vepsi}{{\varepsilon}}
\def\hnorm#1#2{\vert\,#1\,\vert_{#2}}
\newcommand{\R}{{\mathbb R}}
\newcommand{\Sph}{{\mathbb S}}
\def\x{\mathbf{x}}
\def\hvec{\overline{\mathbf{h}}}
\def\evec{\overline{\mathbf{e}}}

\newcommand{ \etal}{\mbox{\emph{et al. }}}

\newcommand\vect[1]{\mbf{#1}}
\newcommand{\mbf}[1]{\mbox{\boldmath$#1$}} 
\newcommand{\RC}[1]{#1 $\times$ #1 $\times$ #1}
\def\um{$\mu$m}
\def\C{$^{\circ}\mathrm{C}$}

\newcommand{\Rmnum}[1]{\expandafter\@slowromancap\romannumeral #1@}

% DEFINITION OF CUSTOM FONT SIZE
\newcommand{\customfontA}{\fontsize{50}{55}\selectfont}
\newcommand{\customfontB}{\fontsize{14.4}{20}\selectfont}
\newcommand{\customfontC}{\fontsize{30}{35}\selectfont}

\DeclareMathAlphabet{\mathpzc}{OT1}{pzc}{m}{it}

\def\clovek#1{\noindent\bgroup\vbox{\noindent#1}\egroup\vskip1em}





% TO INPUT BACKGROUND IMAGE
%\usepackage{eso-pic}
%\newcommand\BackgroundPic{
%\put(0,0){
%\parbox[b][\paperheight]{\paperwidth}{
%\vfill
%\centering
%\includegraphics[width=\paperwidth,height=\paperheight]{img/karel-frontpage.png}
%%\includegraphics[width=\paperwidth,height=\paperheight]{img/background.jpg}
%\vfill
%}}}

\usepackage{fancyvrb}

\newenvironment{bluecode}{\VerbatimEnvironment \color{blue} \begin{Verbatim}}
{\end{Verbatim}}
\newenvironment{greencode}{\VerbatimEnvironment \color{ForestGreen} \begin{Verbatim}}
{\end{Verbatim}}
\newenvironment{redcode}{\VerbatimEnvironment \color{Red} \begin{Verbatim}}
{\end{Verbatim}}

% For Pygments:
\usepackage{fancyvrb}
\usepackage{color}
\usepackage[utf-8]{inputenc}

\makeatletter
\def\PY@reset{\let\PY@it=\relax \let\PY@bf=\relax%
    \let\PY@ul=\relax \let\PY@tc=\relax%
    \let\PY@bc=\relax \let\PY@ff=\relax}
\def\PY@tok#1{\csname PY@tok@#1\endcsname}
\def\PY@toks#1+{\ifx\relax#1\empty\else%
    \PY@tok{#1}\expandafter\PY@toks\fi}
\def\PY@do#1{\PY@bc{\PY@tc{\PY@ul{%
    \PY@it{\PY@bf{\PY@ff{#1}}}}}}}
\def\PY#1#2{\PY@reset\PY@toks#1+\relax+\PY@do{#2}}

\def\PY@tok@gd{\def\PY@tc##1{\textcolor[rgb]{0.63,0.00,0.00}{##1}}}
\def\PY@tok@gu{\let\PY@bf=\textbf\def\PY@tc##1{\textcolor[rgb]{0.50,0.00,0.50}{##1}}}
\def\PY@tok@gt{\def\PY@tc##1{\textcolor[rgb]{0.00,0.25,0.82}{##1}}}
\def\PY@tok@gs{\let\PY@bf=\textbf}
\def\PY@tok@gr{\def\PY@tc##1{\textcolor[rgb]{1.00,0.00,0.00}{##1}}}
\def\PY@tok@cm{\let\PY@it=\textit\def\PY@tc##1{\textcolor[rgb]{0.25,0.50,0.50}{##1}}}
\def\PY@tok@vg{\def\PY@tc##1{\textcolor[rgb]{0.10,0.09,0.49}{##1}}}
\def\PY@tok@m{\def\PY@tc##1{\textcolor[rgb]{0.40,0.40,0.40}{##1}}}
\def\PY@tok@mh{\def\PY@tc##1{\textcolor[rgb]{0.40,0.40,0.40}{##1}}}
\def\PY@tok@go{\def\PY@tc##1{\textcolor[rgb]{0.50,0.50,0.50}{##1}}}
\def\PY@tok@ge{\let\PY@it=\textit}
\def\PY@tok@vc{\def\PY@tc##1{\textcolor[rgb]{0.10,0.09,0.49}{##1}}}
\def\PY@tok@il{\def\PY@tc##1{\textcolor[rgb]{0.40,0.40,0.40}{##1}}}
\def\PY@tok@cs{\let\PY@it=\textit\def\PY@tc##1{\textcolor[rgb]{0.25,0.50,0.50}{##1}}}
\def\PY@tok@cp{\def\PY@tc##1{\textcolor[rgb]{0.74,0.48,0.00}{##1}}}
\def\PY@tok@gi{\def\PY@tc##1{\textcolor[rgb]{0.00,0.63,0.00}{##1}}}
\def\PY@tok@gh{\let\PY@bf=\textbf\def\PY@tc##1{\textcolor[rgb]{0.00,0.00,0.50}{##1}}}
\def\PY@tok@ni{\let\PY@bf=\textbf\def\PY@tc##1{\textcolor[rgb]{0.60,0.60,0.60}{##1}}}
\def\PY@tok@nl{\def\PY@tc##1{\textcolor[rgb]{0.63,0.63,0.00}{##1}}}
\def\PY@tok@nn{\let\PY@bf=\textbf\def\PY@tc##1{\textcolor[rgb]{0.00,0.00,1.00}{##1}}}
\def\PY@tok@no{\def\PY@tc##1{\textcolor[rgb]{0.53,0.00,0.00}{##1}}}
\def\PY@tok@na{\def\PY@tc##1{\textcolor[rgb]{0.49,0.56,0.16}{##1}}}
\def\PY@tok@nb{\def\PY@tc##1{\textcolor[rgb]{0.00,0.50,0.00}{##1}}}
\def\PY@tok@nc{\let\PY@bf=\textbf\def\PY@tc##1{\textcolor[rgb]{0.00,0.00,1.00}{##1}}}
\def\PY@tok@nd{\def\PY@tc##1{\textcolor[rgb]{0.67,0.13,1.00}{##1}}}
\def\PY@tok@ne{\let\PY@bf=\textbf\def\PY@tc##1{\textcolor[rgb]{0.82,0.25,0.23}{##1}}}
\def\PY@tok@nf{\def\PY@tc##1{\textcolor[rgb]{0.00,0.00,1.00}{##1}}}
\def\PY@tok@si{\let\PY@bf=\textbf\def\PY@tc##1{\textcolor[rgb]{0.73,0.40,0.53}{##1}}}
\def\PY@tok@s2{\def\PY@tc##1{\textcolor[rgb]{0.73,0.13,0.13}{##1}}}
\def\PY@tok@vi{\def\PY@tc##1{\textcolor[rgb]{0.10,0.09,0.49}{##1}}}
\def\PY@tok@nt{\let\PY@bf=\textbf\def\PY@tc##1{\textcolor[rgb]{0.00,0.50,0.00}{##1}}}
\def\PY@tok@nv{\def\PY@tc##1{\textcolor[rgb]{0.10,0.09,0.49}{##1}}}
\def\PY@tok@s1{\def\PY@tc##1{\textcolor[rgb]{0.73,0.13,0.13}{##1}}}
\def\PY@tok@sh{\def\PY@tc##1{\textcolor[rgb]{0.73,0.13,0.13}{##1}}}
\def\PY@tok@sc{\def\PY@tc##1{\textcolor[rgb]{0.73,0.13,0.13}{##1}}}
\def\PY@tok@sx{\def\PY@tc##1{\textcolor[rgb]{0.00,0.50,0.00}{##1}}}
\def\PY@tok@bp{\def\PY@tc##1{\textcolor[rgb]{0.00,0.50,0.00}{##1}}}
\def\PY@tok@c1{\let\PY@it=\textit\def\PY@tc##1{\textcolor[rgb]{0.25,0.50,0.50}{##1}}}
\def\PY@tok@kc{\let\PY@bf=\textbf\def\PY@tc##1{\textcolor[rgb]{0.00,0.50,0.00}{##1}}}
\def\PY@tok@c{\let\PY@it=\textit\def\PY@tc##1{\textcolor[rgb]{0.25,0.50,0.50}{##1}}}
\def\PY@tok@mf{\def\PY@tc##1{\textcolor[rgb]{0.40,0.40,0.40}{##1}}}
\def\PY@tok@err{\def\PY@bc##1{\fcolorbox[rgb]{1.00,0.00,0.00}{1,1,1}{##1}}}
\def\PY@tok@kd{\let\PY@bf=\textbf\def\PY@tc##1{\textcolor[rgb]{0.00,0.50,0.00}{##1}}}
\def\PY@tok@ss{\def\PY@tc##1{\textcolor[rgb]{0.10,0.09,0.49}{##1}}}
\def\PY@tok@sr{\def\PY@tc##1{\textcolor[rgb]{0.73,0.40,0.53}{##1}}}
\def\PY@tok@mo{\def\PY@tc##1{\textcolor[rgb]{0.40,0.40,0.40}{##1}}}
\def\PY@tok@kn{\let\PY@bf=\textbf\def\PY@tc##1{\textcolor[rgb]{0.00,0.50,0.00}{##1}}}
\def\PY@tok@mi{\def\PY@tc##1{\textcolor[rgb]{0.40,0.40,0.40}{##1}}}
\def\PY@tok@gp{\let\PY@bf=\textbf\def\PY@tc##1{\textcolor[rgb]{0.00,0.00,0.50}{##1}}}
\def\PY@tok@o{\def\PY@tc##1{\textcolor[rgb]{0.40,0.40,0.40}{##1}}}
\def\PY@tok@kr{\let\PY@bf=\textbf\def\PY@tc##1{\textcolor[rgb]{0.00,0.50,0.00}{##1}}}
\def\PY@tok@s{\def\PY@tc##1{\textcolor[rgb]{0.73,0.13,0.13}{##1}}}
\def\PY@tok@kp{\def\PY@tc##1{\textcolor[rgb]{0.00,0.50,0.00}{##1}}}
\def\PY@tok@w{\def\PY@tc##1{\textcolor[rgb]{0.73,0.73,0.73}{##1}}}
\def\PY@tok@kt{\def\PY@tc##1{\textcolor[rgb]{0.69,0.00,0.25}{##1}}}
\def\PY@tok@ow{\let\PY@bf=\textbf\def\PY@tc##1{\textcolor[rgb]{0.67,0.13,1.00}{##1}}}
\def\PY@tok@sb{\def\PY@tc##1{\textcolor[rgb]{0.73,0.13,0.13}{##1}}}
\def\PY@tok@k{\let\PY@bf=\textbf\def\PY@tc##1{\textcolor[rgb]{0.00,0.50,0.00}{##1}}}
\def\PY@tok@se{\let\PY@bf=\textbf\def\PY@tc##1{\textcolor[rgb]{0.73,0.40,0.13}{##1}}}
\def\PY@tok@sd{\let\PY@it=\textit\def\PY@tc##1{\textcolor[rgb]{0.73,0.13,0.13}{##1}}}

\def\PYZbs{\char`\\}
\def\PYZus{\char`\_}
\def\PYZob{\char`\{}
\def\PYZcb{\char`\}}
\def\PYZca{\char`\^}
\def\PYZsh{\char`\#}
\def\PYZpc{\char`\%}
\def\PYZdl{\char`\$}
\def\PYZti{\char`\~}
% for compatibility with earlier versions
\def\PYZat{@}
\def\PYZlb{[}
\def\PYZrb{]}
\makeatother
% End of Pygments inputs.

% Define color boxes:
\definecolor{MyGreen}{rgb}{0.9, 1, 0.9}
\makeatletter\newenvironment{gbox}{%
   \begin{lrbox}{\@tempboxa}\begin{minipage}{0.985\columnwidth}}{\end{minipage}\end{lrbox}%
   \noindent
   \colorbox{MyGreen}{\usebox{\@tempboxa}}
}\makeatother

%\definecolor{MyYellow}{rgb}{0.98, 0.98, 0.824}
\definecolor{MyYellow}{rgb}{1, 0.99, 0.8}
\makeatletter\newenvironment{ybox}{%
   \begin{lrbox}{\@tempboxa}\begin{minipage}{0.985\columnwidth}}
   {\end{minipage}\end{lrbox}%
   \noindent
   \colorbox{MyYellow}{\usebox{\@tempboxa}}
}\makeatother

\definecolor{MyBlue}{rgb}{0.88, 0.95, 1}
\makeatletter\newenvironment{bbox}{%
   \begin{lrbox}{\@tempboxa}\begin{minipage}{0.985\columnwidth}}
   {\end{minipage}\end{lrbox}%
   \noindent
   \colorbox{MyBlue}{\usebox{\@tempboxa}}
}\makeatother

\definecolor{MyBlue}{rgb}{0.88, 0.95, 1}
\makeatletter\newenvironment{bboxshort}{%
   \begin{lrbox}{\@tempboxa}\begin{minipage}{0.95\columnwidth}}
   {\end{minipage}\end{lrbox}%
   \noindent
   \colorbox{MyBlue}{\usebox{\@tempboxa}}
}\makeatother


\usepackage{wallpaper}


\begin{document}


%%%%%%%%%%%%%%%%%%%%%%%%%%%%%%%%%%%%%%%%%%%%%%%%%%%%%%%%%%%%%%%%%%%%%%%%%
\pagestyle{empty}

\ThisULCornerWallPaper{1.02}{img/karel-cover.png}






%%%%%%%%%%%%%%%%%%%%%%%%%%%%%%%%%%%%%%%%%%%%%%%%%%%%%%%%%%%%%%%%%%%%%%%%%
\newpage
\vbox{}
\newpage
\vbox{}
\vfill

\centerline{Revision April-11-2013}

%%%%%%%%%%%%%%%%%%%%%%%%%%%%%%%%%%%%%%%%%%%%%%%%%%%%%%%%%%%%%%%%%%%%%%%%%
\newpage

\noindent
{\bf About this Textbook}\\[2mm]
Karel the Robot is a widely used programming language that was created at the Stanford University. 
This free open source Karel textbook is provided as a courtesy to NCLab users. It will help you develop 
{\em algorithmic thinking} which is the most important skill in computer programming. Algorithmic 
thinking is ability to translate ideas into procedures, or sequences of simple steps, that are
compatible with how the computer operates. Once you are able to "think like a computer", learning 
any new computer language becomes a matter of merely learning the new syntax.This course should 
not take you more than a few days, and after that you will be able to transition swiftly to Python 
or any other programming language of your choice. \\[2mm]

\noindent
{\bf Become a Co-Author}\\[2mm]
We do not plan to publish the textbook with a commercial publisher since this 
would make it unnecessarily expensive for kids and students who are the main 
target audience. Feel free to contribute to the textbook with any material or 
suggestions. There is never enough illustrations and exercises, and there always 
are bugs to report. Translating the textbook into other languages would benefit
thousands of kids worldwide. Instructions for contributors can be found 
below.\\[2mm]

\noindent
{\bf How to Contribute (for \LaTeX \ and Git users)}\\[2mm]
\noindent
The textbook is written in \LaTeX, a high-quality typesetting system that 
you can learn and use in NCLab. In the future it will be possible to contribute to 
the textbook directly in NCLab, but at this time, the sources are stored 
in a public Git repository {\tt nclab-textbook-karel} at Github (http://github.com). \\[2mm]

\noindent
{\bf How to Contribute (for all others)}\\[2mm]
\noindent
We will gladly accept new interesting exercises for Karel and/or Python, as well as
new images of good quality that will make the textbook more interesting and fun. You
can send those at any time via email to {\tt pavel@femhub.com}.\\[2mm]

\noindent
{\bf List of Contributors}
\begin{itemize}
\item Pavel Solin, University of Nevada, Reno (primary author). 
\item Martin Novak, Czech Technical University, Prague, Czech Republic.
\item Salih Dede, Coral Academy of Science High School, Reno, NV.
\item Nazhmiddin Shapoatov, Sonoran Science Academy, Phoenix, AZ.
\item Veronica McVey, Winnemucca Junior High School, NV.
\item Jozsef Hollosi, Westfield, NJ.
\item Ed Keppelmann, Reno, NV.
\end{itemize}
\vspace{2mm}
\noindent
{\bf Graphics Design:} TR-Design {\tt http://tr-design.cz}

%\noindent
%{\bf For Instructors}\\[4mm]
%Review Book and Exercise Book containing 
%review questions with answers and programming exercises with
%solutions, are part of the NCLab-powered course 
%{\em Intro to Programming with Karel the Robot and Python} that is 
%available at \\[4mm]
%
%{\color{blue}
%\centerline{\tt http://introtoprogramming.net}
%}
%\vspace{5mm}
%
%\noindent
%for a small subscription fee. The fee is used to cover cloud computing resources, 
%development, maintenance, and user support. 
%
%The course is completely web-browser based, no installation of anything at your school 
%or home is needed. You and your students can access the course from anywhere and at any 
%time. Instructor's workflow includes downloading assignments and review
%question worksheets from the database, sending them to the students 
%via one mouse click, and collecting them back, automatically graded. 
%The course is scheduled to open in January 2013. In the meantime, enjoy 
%Karel and Python, and let us know with any questions at {\tt support@nclab.com}!


\newpage
%{\ }
\setcounter{tocdepth}{2}
\tableofcontents
%\pagestyle{plain}

\newpage

\pagestyle{plain}
\setcounter{page}{1}

%%%%%%%%%%%%%%%%%%%%%%%%%%%%%%%%%%%%%%%%%%%%%%%%%%%%%%%%%%%%%%%%%%%%%%%%%
\pagestyle{plain}
\setcounter{page}{1}
\section*{Foreword}

This course provides an efficient introduction to modern algorithmic design and computer programming
using the educational programming language {\em Karel the Robot}. The language was 
created by Richard E. Pattis at Stanford University in his famous book {\em Karel the Robot: 
A Gentle Introduction to the Art of Programming}. With Karel, learning feels more like playing. 
You will not be exposed to technical difficulties of conventional programming languages, and 
your learning will not be obscurred by math. As a result, you will understand quickly and easily 
what computer programming really is about. After Karel, picking up additional programming 
languages will be much easier.

We have talked to many students who struggled with computer programming. The most frequent cause 
for that was a mistake of their instructor who believed that "real programming skills 
are best taught with a real programming language". Most often they were using C/C++ or Java. This, 
however, is the same mistake as teaching someone how to ride a motorcycle without showing him or 
her how to ride a bicycle first. There is nothing wrong with motorcycles of course! 

\begin{figure}[!ht]
\begin{center}
\includegraphics[width=0.75\textwidth]{img/fore-1.png}
\end{center}
\vspace{-2mm}
\caption{Sample game {\em Diamond Mine} in Manual mode.}
\label{fig:f1}
\vspace{-4mm}
\end{figure}
\noindent
It is exactly the same thing with programming -- as a matter of fact, the "real" programming 
languages have more complicated syntax, meaning that it takes a lot of work to express even the simplest ideas. 
It also takes a lot of work to produce any graphics. As a result, students are often exposed
to programming math formulas in one way or another. These things together create a frustration with 
computer programming which is absolutely unnecessary.

In contrast to that, Karel the Robot is an outstanding programming language for beginners. The robot only knows a handful of 
simple commands and has a few sensors to navigate through the maze. There is no mathematics whatsoever,
and writing programs is as simple as "while not wall go". 

The course begins in Manual Mode where the robot can be guided via 
clicking on five buttons {\em Go} (make one step forward), {\em Left} (turn left), {\em Right} 
(turn right), {\em Put} (put a gem on the groud) and {\em Get} (pick up a gem from the ground). 
In the next Programming Mode 1 a.k.a ``Bridge to Programming'' students keep solving 
problems by typing the commands {\tt go}, {\tt left}, {\tt right}, {\tt put} and {\tt get} 
instead of clicking on buttons. This Mode also contains the counting loop ({\tt repeat} command). 

The need for higher functionality such as conditional loops, conditions, and custom commands 
arises naturally as game goals become more complicated. 
Students learn quickly that it is advantageous to break complex tasks into smaller 
ones, which is one of the most important principle of computer 
programming. The textbook is written by programming experts, and in addition to 
advanced programming skills the students gain an overview of good and bad 
programming habits.

\begin{figure}[!ht]
\begin{center}
\includegraphics[width=0.75\textwidth]{img/fore-2.png}
\end{center}
\vspace{-2mm}
\caption{Sample game {\em Speleologist} in Programming mode.}
\label{fig:f2}
\vspace{-4mm}
\end{figure}
\noindent


\part{Meet Karel}

\input textbook.tex

\part{Programming Exercises}
\setcounter{section}{0}

\input exercises.tex

\part{Review Questions}
\setcounter{section}{0}

\input review-questions.tex

\end{document}
